%Group Talk 25/5/2014 Draft 1
%Talk script
\documentclass{article}
\usepackage{times}
\usepackage{amsmaths}
\title{Group Talk Script}
\author{Ciaran Cruise, Alexander Garrett}
\date{\today}
\begin{document}
\maketitle



INTRODUCTION
[SLIDE 1 - Title Slide]\\
Opening: 
- Demonstration of the software, which emphasis on showings its capacity and modularity (getting people to chose values)

WHAT DOES PROJECT THALES MEAN TO YOU? - THE NEED
We take complex situations that can be modeled using the Laplace equation (this includes not only electrostatics, but fluid dynamics, and thermal emission)
In an ideal world even a complex situation could be solved analytically, however sometimes its just not possible and if it is, even a deciptively simple
looking situation can take up to 4 weeks (and used tissues) to solve - trust us, we know. So how can we sidestep this? 
-----------------------
[SLIDE 3 - The Task]
How we did it:
This is where Project Thales comes to the fore. We take your complex situation and reduce it something as simple, elegant the laplace equation itself.  
Project THALES is software package that provides a fast, accurate and user friendly solution to situations which use the Laplace equation. It is based
on a number of solid design principles: namely, MODULARITY/MODULAR DESIGN, INTUITIVE DESIGN and POWERFUL COMPONENT BASE - look to change- all of which 
unite under the banner of an adaptable and versatile piece of software. \\
------------------------
[SLIDE 4 - Main Message]
- slot for demonstration -- look, we did it\\
Goals of the demonstration:
RUN THE PROGRAM
TAKE USER VALUES
Model a given boundary condition situation
Reinforce throughout the concept of user-definablity, the programs flexibility, ease of data communication/INPUT/OUTPUT - including export,
and at no point is the underlying physics compromised nor the user, particularily, if the user is non computer literate. 
-----------------------

[SLIDE 5 - Review]
CONTENTS
- Discussion on the development cycle
The importance of accuracy: solving the analytical solution
The main purpose: solving situations numerically
Putting pen to paper: creating the software package\\
\begin{center}
\textbf{MAIN BODY}\\
\end{center}

[SLIDE 6 - THE RUN DOWN]
[SLIDE 7 - GRAPHIC OF ANALYTICAL SOLUTION]
Solving the analytical solution:\\
Central to benchmarking the accuracy of our ability to solve a given situation, was a comparison between an analyitcal situation with a numerical
one calculated by the software. So you can see here, the situation which we solved analytically - the infite cylinder over a grounded plate. This
was solved using a combination of Legandre polynomaials and an augmented method of images - for further details on this, see our published 
group report, rest assured it is a deciptively difficult peice of mathematics. Comparison between analytical and numerical [see here in future 
for specifics].
[SLIDE 8 - THE NUMERICAL SOLUTION]
[SLIDE 9 - GRAPHIC OF ENGINE TESTING]
The Numerical:\\
Our capacity to solve situations numerically is the corner stone of our program; if this fails then so does our entire program - its very fortunate then
that Project THALES has a robust and stable backbone. You can see also that we have direct comparison with other exisiting finite difference engines,
notably the method succesive over relaxation by our very own, Mr A.M. Q. DeGarrett. [Being that SOR is (hopefully marginally) faster, yet less 
sophsticated we can directly show you why we have an edge over competeing products.] Furthermore we one again emphesise the modular component of the 
engine and hence the project as a whole. Just as we have reduced the complexity of electrostatic problems to simple, detailed outputs of the
underlying physics , we can also do similarly for fluid dynamics, thermal emissons, any laplace situation, by simply changing a few lines of code
and a few constants.

[SLIDE 10 - THE SOFTWARE LAYOUT AND PACKAGE]
[SLIDE 11 - GRAPHICS OF THE UI]
One of the major considerations of project THALES is who its target audience is, and, at first it may seem completely obvious: of course, its
undergraduates, postgraduates and those in industry - however, we feel that project THALES also has a great potential to be used by.\
We felt that considering we had this powerful,   


\begin{center}
\textbf{CONCLUSION}\\
\end{center}


[SLIDE 7 - 2ND REVIEW -RRR-]
[SLIDE 8 - CONCLUSION]
Include what the future of the project may entail.
[SLIDE 9 - END TITLE PAGE]
-----------------
[SLIDE 10 - DOCUMENTATION LINKS]
[SLIDE 11 - FAQ]

\end{document}

%HIDDEN SECTIONS
%As you have seen from this demonstration, the  principles which we mentioned before, - RRR - , are immeadiately visible when using the program, 
%however let us run through, in more detail, what we mean:
%Powerful Component base: The engine\\
%The engine is a finite difference engine, however, it is a rather sophisticated one, termed a a multigrid method. Multigird methods have significant 
%advantages in terms of speed, accuaracy and programming complexity over other finite differencing methods. Given then teams significant background in 
%differencing engines, to attempt something new and bold , seemed a logical progression. 
%Modularity: 
